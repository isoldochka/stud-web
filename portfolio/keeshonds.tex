
\documentclass{article}
\usepackage[utf8]{inputenc} 
\usepackage[russian]{babel}
\title{О кеесхондах}
\author{Вилков Сергей Владимирович}
\date{11/2013}
\begin{document}
  \maketitle
  Мы рады приветствовать Вас на сайте нашего питомника собак
  породы кеесхондов!

  Если Вы оказались на нашей страничке, наверняка это
  неслучайно. Мы постараемся не только рассказать об этой
  удивительной породе собак, но и сделать так, чтобы Вы смогли
  найти настоящего верного друга, который станет поистине
  неотъемлемой частью Вашей семьи и подарит целое море
  положительных эмоций детям и взрослым.

  Кто же такие кеесхонды, или немецкие шпицы? Прежде всего, это
  собака-компаньон, которая сочетает в себе качества чуткого
  сторожа и идеальной «семейной» собаки. Неслучайно многие
  психотерапевты используют немецких шпицев в программах
  по работе с замкнутыми пациентами – дружелюбные,
  коммуникабельные, ласковые и добродушные собаки легко
  находят контакт с людьми и помогают снять напряжение, они
  всегда чутко воспринимают настроение хозяина. У кеесхондов,
  неспроста называемых «улыбающимися голландцами», совершенно
  отсутствует проявление агрессии в поведении. Они не трусливы,
  не капризны и не своенравны.

  Конечно же, каждая собака уникальна и имеет свой неповторимый
  характер и нрав. Но все же оставаться равнодушным при общении
  с кеесхондом просто невозможно – это сообразительные,
  смелые и бесконечно верные питомцы, заражающие своей
  жизнерадостностью. Шпицы легко уживаются с другими домашними
  животными, перенимают режим дня своих хозяев и прекрасно
  поддаются дрессировке. Поэтому даже если Ваш питомец будет
  проявлять излишнюю активность или самостоятельность на
  прогулке, существуют достаточно простые приемы воспитания
  любимца. Кеесхонды легко обучаются, они старательны и очень
  послушны.

  Стоит ли заводить кеесхонда, если в семье есть
  ребенок? Безусловно – шпицы очень активны и любят длительные
  прогулки и «купание» в густой траве и пушистом снегу,
  развлечения на воде. Они являются замечательными психологами
  для детей, помогающими наладить контакт с окружающим миром и
  подарить целое море радости и незабываемых эмоций для ребенка.

  Немецкие шпицы легко адаптируются к жизни в городской
  квартире при условии регулярного выгула и умеренного
  количества активных упражнений, игр и тренировок. Они не любят
  оставаться одни, лучшее времяпровождения для них – вместе
  со своими хозяевами. Кеесхонды великолепно реализуют свои
  способности при занятиях аджилити, где они могут проявить свою
  сообразительность, ловкость, прыгучесть и навыки, полученные
  во время дрессировки

  Кеесхондам от предков достался сильный сторожевой
  инстинкт. Характерный звонкий лай и внушительный вид животного
  способен отпугнуть злоумышленников. Но на деле, конечно, это
  максимальная «угроза», исходящая от питомцев – как говорится,
  «лает, но не кусает». Немецкий шпиц является одной из старейших
  пород собак, пришедшей к нам из Германии и Голландии XVI века, и
  наименее подвергнувшейся искусственному отбору на протяжении
  веков.

  Кеесхонды гармонично сложены, средним ростом до полуметра. Они
  отличаются крепким здоровьем и выносливостью. При правильном
  и грамотном уходе собаки практически не болеют. Забота о
  вольфшпице не потребует от Вас особых усилий – его густая
  роскошная шерсть серебристого оттенка за счет специфики
  подшерстка не сильно осыпается даже в периоды линьки. Волос
  практически не скатывается в колтуны, еженедельного
  вычесывания будет достаточно для поддержания шерсти в
  отличном состоянии (во время линьки – несколько раз в неделю,
  что сокращает период линьки до четырех недель). Более частое же
  расчесывание наоборот может привести к удалению подшерстка,
  что чревато ухудшением внешнего вида шерсти. Поэтому вопреки
  частому мнению о собаках с длинной шерстью, немецкий шпиц
  доставляет своим хозяев в этом отношении гораздо меньше забот.

  Окрас шерсти кеесхонда – волчий, со смесью серого и черного и
  характерными «очками» вокруг темно-коричневых миндалевидных
  глаз. На мордочке – характерная маска черного окраса. Кончик
  хвоста и маленькие стоячие ушки также черного цвета.

  Немецкий шпиц линяет два раза за год, весной и осенью. К лету
  собака сбрасывает часть шерсти, что помогает ей легче перенести
  жару. Но ни в коем случае нельзя дополнительно стричь животное,
  чтобы не подвергать кожу излишнему воздействию солнца.

  Конечно же, стоит помнить о том, что, как и любому живому
  существу, кеесхонду хочется получать от хозяев ласку, тепло и
  заботу. Энергичный и подвижный любитель игр на свежем воздухе
  лучше всего подойдет семье с детьми или ведущей активный
  образ жизни. В то же время, трудно подобрать более ласкового
  компаньона для пожилого человека. Принимая в свою семью
  пушистое чудо с лисьей мордочкой и веселым нравом, Вы навсегда
  получите преданного друга, надежного спутника и бесконечно
  преданного хранителя домашнего очага, для которого Вы станете
  целым миром.
\end{document}
