\documentclass{article}
\usepackage[utf8]{inputenc} 
\usepackage[russian]{babel}
\title{О подборе щенка}
\author{Вилков Сергей Владимирович}
\date{11/2013}
\begin{document}
  \maketitle
  Итак, у Вас наконец-то появилась возможность принять в семью
  четвероного друга. Выбор ваш пал на замечательных роскошных
  кеесхондов. И тут возникает важный вопрос – выбор пола щенка.

  Безусловно, принимать решение при выборе пола щенка
  нужно взвешенно и обдуманно, тщательно взвесив все плюсы и
  минусы. Выбор пола собаки стоит предоставить члену семьи, на
  которого в основном лягут обязанности по уходу за ней. Принимая
  решение, необходимо учесть основные особенности характера и
  физиологии кобелей и сук и определиться, для чего Вы заводите
  щенка. Будет ли Ваш любимец участвовать в выставках, собираетесь
  ли Вы заниматься племенным разведением или же Ваше намерение
  завести собаку обусловлено только лишь желанием приобрести
  верного товарища и домашнего любимца.

  Если Вы решили завести очаровательного благородного кеесхонда
  для последующего разведения, конечно же, Вы должны быть готовы
  взять на себя ответственность за период беременности суки и за
  растущих щенков, за их благополучное пристройство. Разведение
  потомства – это большой серьезный шаг для владельца.

  Примерно каждые полгода владелец суки сталкивается с таким
  явлением, как течка, который длится в среднем 21-28 дней. Часто
  требуется дополнительный уход за животным во время течки, но и
  тут все индивидуально. Наша Дива, к примеру, очень чистоплотна,
  и мы обходимся без использования специальных трусов.

  Найти достойного кобеля для вязки гораздо сложнее – к ним
  предъявляются более высокие требования. Здесь недостаточно
  одной лишь безупречной родословной или выдающегося экстерьера.

  Периодичность линьки у сук – обычно два раза в год. После родов
  собака также теряет шерсть, максимальная потеря - примерно
  через три месяца после родов. Через 7 месяцев шерсть приобретает
  прежнюю густоту, которая держится до следующих возможных родов
  (примерно полтора года). Кобели линяют один раз в год.

  В вопросах выставок кобели более приспособлены, хозяева меньше
  привязаны к периодичности линьки. Хотя просчитать регулярность
  линьки сук также несложно. К тому же, в юниорах суки выглядят
  более сформированными, «одетыми». Поэтому необязательно
  торопиться с вязкой собаки, чтобы избежать послеродовой линьки,
  до 2-3 лет выступая и получая титулы. Кобели же в подростковом
  возрасте еще не до конца сформированы.

  Что касается внешних различий кеесхондов, следует отметить,
  что породный тип ярче выражен у кобелей. Кобели боле массивны,
  тяжелы и мужественны. Внешний вид ярче, нарядней и выразительней,
  нежели чем у кеесхонда-девочки. У кобелей воротничок на шерсти
  и груди больше напоминает пышную густую гриву. Суки выглядят
  «помельче», шерсть их не такая обильная. Их головы аккуратные,
  точеные, внешний вид изящный и уточненный, одним словом –
  женственный.

  Так или иначе, все владельцы собак любой породы сталкиваются
  с половыми различиями питомцев в части внешних данных и
  физиологических особенностей. Но, наверное, Вы согласитесь,
  что для хозяина не меньшую важность имеет характер собаки, в
  некоторый степени также зависящий от пола.

  В возрасте до года щенки кеесхонда активные, игривые, и
  между ними не наблюдается принципиальной разницы. Период
  полового созревания также может не сильно отличаться у собак
  обоих полов. Становясь взрослее, суки приобретают мягкость и
  рассудительность характера, благоразумность. И, вместе с тем,
  эти пушистые красавицы хитры, умны и сообразительны, и могут
  проявлять недюжинную настойчивость и изобретательность,
  чтобы подластиться к хозяину и добиться своего. Повод для
  беспокойства у хозяина появляется лишь в период течки, когда
  сука выгуливается исключительно на поводке для избегания
  нежелательной вязки. Кобели немецкого шпица имеют более
  простодушный, прямолинейный, игривый и открытый характер, это
  вечные дети, готовые бесконечно привлекать внимание любимого
  хозяина. Хотя и здесь есть свои особенности, когда кобель бывает
  крайне неосторожен на прогулке при виде суки.

  В независимости от того, какого пола Ваш домашний любимец,
  ему, безусловно, необходимо получать от своих хозяев любовь
  и заботу. Регулярные прогулки, игры, активные развлечения
  вместе с Вами сделают его счастливым и жизнерадостным. Каждая
  собака неповторима и уникальна, и, несмотря на существенные
  различия, обусловленные половой принадлежностью, имеет
  свой индивидуальный характер. И даже если Вам сложно отдать
  предпочтение при выборе щенка – мальчика или девочки
  – выбирайте того, к кому будет лежать Ваше сердце. В любом
  случае Вы получаете море любви, преданности и общения. А по мере
  взросления любимца Вы приспособитесь ко всем нюансам, и тогда
  будет казаться, что выбор Ваш стал единственно правильным!
\end{document}
